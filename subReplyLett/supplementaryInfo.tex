\documentclass[aps,pre,superscriptaddress,floats,11pt]{revtex4}
% \documentclass[aps,pre,superscriptaddress,11pt]{revtex4-1}
\usepackage{amsfonts,amssymb,amsmath,times}
\usepackage{graphicx,multirow}
\usepackage{bm}
\usepackage{enumerate}
\usepackage{color}
\usepackage{pgffor}
\usepackage{ifthen}
\usepackage{morefloats}
\usepackage[floats]{preview}
% \usepackage{longtable,lscape,subfigure}
\usepackage{array} % for extrarowheight
\setlength{\extrarowheight}{1.5pt}
% \usepackage{placeins}
 
\linespread{1.5}

\begin{document}

\title{Supplementary Materials: \\ Ordinal partition transition network based complexity measures for inferring coupling direction and delay from time series}

%\date{\today}

\maketitle

% {\bf{SI Text: }}
\renewcommand{\thepage}{SM-\arabic{page}}  
\renewcommand{\thesection}{SM-\Roman{section}}   
\renewcommand{\theequation}{S\arabic{equation}}  
\renewcommand{\thetable}{S\arabic{table}}   
\renewcommand{\thefigure}{S\arabic{figure}}

\begin{figure}[htb]
	\centering
	\includegraphics[width=0.7\columnwidth]{/Users/yongzou/Documents/TimeSeries_Network/OrderPatternPermutation/RuanFiguresPlot2018/rev_Figures/sigma_lengthBCDE.eps}
\caption{Double logarithmic plot of the dependence of $\sigma_{X\to Y}$ on the sample size $N$ for the optimal (causal) lags (blue/red) and some non-causal lag (black) for the four cases of coupled linear-stochastic systems studied in the main paper: (a) Eq.~(1) (unidirectional), (b) Eq.~(9) (unidirectional), (c) Eq.~(10) (symmetric bidirectional), (d) Eq.~(11) (asymmetric bidirectional). In (c,d), the values for both causal delays are shown. Error bars indicate the associated standard deviation (in linear scale) over 20 independent realizations.  \label{fig:sampleSizeBCDEsigma}}
\end{figure}

\begin{figure}[htb]
	\centering
	\includegraphics[width=0.7\columnwidth]{/Users/yongzou/Documents/TimeSeries_Network/OrderPatternPermutation/RuanFiguresPlot2018/rev_Figures/entropyH_lengthBCDE.eps}
\caption{Same as Fig.~\ref{fig:sampleSizeBCDEsigma}, but for the co-occurrence entropy $H_{X\to Y}(\tau)$. Note that for $D=5$, we expect a co-occurrence entropy $H_{X \to Y} \approx \log_2 (5!) \approx 6.9$ for two series of ordinal patterns that are obtained from two independent identically distributed noise processes. Considering this fact, we show the difference between our estimated $H_{X \to Y}$ values and that maximum value of 6.9, which yields similar asymptotic results as those for $\sigma_{X\to Y}$ and KLD. Notably, at the positions of non-causal delays, zero values of the aforesaid difference are expected asymptotically, while non-zero values appear at the positions of causal delays. \label{fig:sampleSizeBCDEentropy}}
\end{figure}

% where / How to put the case of D = 4? 
% sigma
\begin{figure}[htb]
	\centering
	\includegraphics[width=0.7\columnwidth]{/Users/yongzou/Documents/TimeSeries_Network/OrderPatternPermutation/RuanFiguresPlot2018/rev_Figures/d4sigma_lengthBCDE.eps}
\caption{Same as Fig. \ref{fig:sampleSizeBCDEsigma}, but for $D = 4$.   \label{fig:d4sampleSizeBCDEsigma}}
\end{figure}

% H
\begin{figure}[htb]
	\centering
	\includegraphics[width=0.7\columnwidth]{/Users/yongzou/Documents/TimeSeries_Network/OrderPatternPermutation/RuanFiguresPlot2018/rev_Figures/d4entropyH_lengthBCDE.eps}
\caption{Same as Fig.~\ref{fig:sampleSizeBCDEentropy}, but for the $H_{X\to Y}(\tau)$ and $D=4$. Note that we expect a co-occurrence entropy $H_{X \to Y} \approx \log_2 (4!) \approx 4.585$ for two series of ordinal patterns that are obtained from two independent identically distributed noise processes. \label{fig:d4sampleSizeBCDEentropy}}
\end{figure}

% KLD? 
\begin{figure}
	\centering
	\includegraphics[width=0.7\columnwidth]{/Users/yongzou/Documents/TimeSeries_Network/OrderPatternPermutation/RuanFiguresPlot2018/rev_Figures/d4kld_lengthBCDE.eps}
\caption{(Color online) Same as Fig.~\ref{fig:sampleSizeBCDEsigma}, but for the KLD and $D = 4$.  \label{fig:d4sampleSizeBCDEkld}}
\end{figure}

% end of D = 4 stochastic models 


% D = 5 for the Henon maps
\begin{figure}[htb]
	\centering
	\includegraphics[width=0.7\columnwidth]{/Users/yongzou/Documents/TimeSeries_Network/OrderPatternPermutation/RuanFiguresPlot2018/rev_Figures/sigma_lengthHenon.eps}
\caption{Same as in Fig.~\ref{fig:sampleSizeBCDEsigma} but for two coupled H\'enon maps at four different coupling strengths $\mu$: (a) $\mu=0.2$ (b) $0.3$, (c) $0.4$, and (d) $0.5$. The blue lines show the behavior at the delay corresponding to the maximum value of $\sigma_{X\to Y}$, while the black ones show the values for a non-causal delay of $\tau=10$. \label{fig:sampleSizeHenonsigma}}
\end{figure}

\begin{figure}[htb]
	\centering
	\includegraphics[width=0.7\columnwidth]{/Users/yongzou/Documents/TimeSeries_Network/OrderPatternPermutation/RuanFiguresPlot2018/rev_Figures/entropyH_lengthHenon.eps}
\caption{Same as Fig.~\ref{fig:sampleSizeHenonsigma}, but for the co-occurrence entropy $H_{X\to Y}(\tau)$. \label{fig:sampleSizeHenonentropy}}
\end{figure}

% D = 4 for the Henon maps
\begin{figure}[htb]
	\centering
	\includegraphics[width=0.7\columnwidth]{/Users/yongzou/Documents/TimeSeries_Network/OrderPatternPermutation/RuanFiguresPlot2018/rev_Figures/d4sigma_lengthHenon.eps}
\caption{Same as in Fig.~\ref{fig:sampleSizeHenonsigma} but for $\sigma_{X \to Y}$ and $D = 4$. \label{fig:d4sampleSizeHenonsigma}}
\end{figure}

\begin{figure}[htb]
	\centering
	\includegraphics[width=0.7\columnwidth]{/Users/yongzou/Documents/TimeSeries_Network/OrderPatternPermutation/RuanFiguresPlot2018/rev_Figures/d4entropyH_lengthHenon.eps}
\caption{Same as in Fig.~\ref{fig:sampleSizeHenonsigma} but for $H_{X \to Y}$ and $D = 4$. \label{fig:d4sampleSizeHenonentropy}}
\end{figure}

\begin{figure}[htb]
	\centering
	\includegraphics[width=0.7\columnwidth]{/Users/yongzou/Documents/TimeSeries_Network/OrderPatternPermutation/RuanFiguresPlot2018/rev_Figures/d4kld_lengthHenon.eps}
\caption{Same as in Fig.~\ref{fig:sampleSizeHenonsigma} but for KLD and $D = 4$. \label{fig:d4sampleSizeHenonkld}}
\end{figure}

% end of D = 4 


% temperature records
\begin{figure}[htb]
	\centering
	\includegraphics[width=0.7\columnwidth]{/Users/yongzou/Documents/TimeSeries_Network/OrderPatternPermutation/RuanFiguresPlot2018/rev_Figures/sigentropy_lengthTemp.eps}
\caption{Dependence of (a) $\sigma_{X\to Y}$ and (b) $H_{X\to Y}$ on the sample length $N$ at the optimal delay ($\tau=1$ day, blue) and some arbitrarily chosen large delay ($\tau=20$ days, black) for the real-world example of two temperature time series from Oxford ($X$) and Vienna ($Y$) as discussed in the main text. \label{fig:sampleSizeTempentropy}}
\end{figure}


% or we put all cases of D = 4 here? 

\end{document}