\documentclass[aps,pre,superscriptaddress,floats,11pt]{revtex4}
% \documentclass[aps,pre,superscriptaddress,11pt]{revtex4-1}
\usepackage{amsfonts,amssymb,amsmath,times}
\usepackage{graphicx,multirow}
\usepackage{bm}
\usepackage{enumerate}
\usepackage{color}
\usepackage{pgffor}
\usepackage{ifthen}
\usepackage{morefloats}
\usepackage[floats]{preview}
% \usepackage{longtable,lscape,subfigure}
\usepackage{array} % for extrarowheight
\setlength{\extrarowheight}{1.5pt}
% \usepackage{placeins}
 
\linespread{1.5}

\begin{document}

\title{Supplementary Materials: \\ Ordinal partition transition network based complexity measures for inferring coupling direction and delay from time series}

\date{\today}

\maketitle

% {\bf{SI Text: }}
\renewcommand{\thepage}{SM-\arabic{page}}  
\renewcommand{\thesection}{SM-\Roman{section}}   
\renewcommand{\theequation}{S\arabic{equation}}  
\renewcommand{\thetable}{S\arabic{table}}   
\renewcommand{\thefigure}{S\arabic{figure}}


\section{Sample size effects on $\sigma_{X\to Y}$}
\begin{enumerate}
\item Stochastic models (Fig. \ref{fig:sampleSizeBCDEsigma})
\begin{figure}[htb]
	\centering
	\includegraphics[width=0.7\columnwidth]{/Users/yongzou/Documents/TimeSeries_Network/OrderPatternPermutation/RuanFiguresPlot2018/rev_Figures/sigma_lengthBCDE.eps}
\caption{(Color online) Double logarithmic plot of the dependence of $\sigma_{X\to Y}(\tau)$ on the sample size $N$ for the optimal (causal) lags (blue/red) and some non-causal lag (black) for the four cases of coupled linear-stochastic systems: (a) Eq. 1 (unidirectional), (b) Eq. 9 (unidirectional), (c) Eq. 10 (symmetric bidirectional), (d) Eq. 11 (asymmetric bidirectional). In (c,d), the values for both causal delays are shown. Errorbars correspond to the standard deviation (linear scale) over 20 independent realizations.  \label{fig:sampleSizeBCDEsigma}}
\end{figure}

\item Coupled Henon maps (Fig. \ref{fig:sampleSizeHenonsigma})
\begin{figure}[htb]
	\centering
	\includegraphics[width=0.7\columnwidth]{/Users/yongzou/Documents/TimeSeries_Network/OrderPatternPermutation/RuanFiguresPlot2018/rev_Figures/sigma_lengthHenon.eps}
\caption{(Color online) Same as in Fig. \ref{fig:sampleSizeBCDE} but for the two coupled H\'enon maps at four different coupling strengths $\mu$: (a) $\mu=0.2$ (b) $0.3$, (c) $0.4$, and (d) $0.5$. The blue lines show the behavior at the delay corresponding to maximum KLD, while the black ones show the values for a non-causal delay of $\tau=10$. \label{fig:sampleSizeHenonsigma}}
\end{figure}


\end{enumerate}

\section{Sample size effects on co-occurrence entropy $H_{X\to Y}$}
It is expected that the co-occurrence entropy $H_{X \to Y} \approx 6.9$ when $D = 6$ for two series of ordinal patterns that are obtained from two independent identically distributed noise processes. Considering this fact, we calculate the distance between $H_{X \to Y}$ and the expected value of 6.9 ($D = 6$ is used), which yields similar asymptotical results as that of $\sigma_{X\to Y}$ and KLD. Namely, at positions of non-causal delays, zero values are expected asymptotically, while non-zero values appear at the positions of causal delays. 
\begin{enumerate}
\item Stochastic models (Fig. \ref{fig:sampleSizeBCDEentropy})
\begin{figure}[htb]
	\centering
	\includegraphics[width=0.7\columnwidth]{/Users/yongzou/Documents/TimeSeries_Network/OrderPatternPermutation/RuanFiguresPlot2018/rev_Figures/entropyH_lengthBCDE.eps}
\caption{(Color online) The caption is the same as Fig.~\ref{fig:sampleSizeBCDEsigma}, but for the co-occurrence entropy $H_{X\to Y}(\tau)$. \label{fig:sampleSizeBCDEentropy}}
\end{figure}

\item Coupled Henon maps (Fig. \ref{fig:sampleSizeHenonentropy})
\begin{figure}[htb]
	\centering
	\includegraphics[width=0.7\columnwidth]{/Users/yongzou/Documents/TimeSeries_Network/OrderPatternPermutation/RuanFiguresPlot2018/rev_Figures/entropyH_lengthHenon.eps}
\caption{(Color online) The caption is the same as Fig.~\ref{fig:sampleSizeHenonsigma}, but for the co-occurrence entropy $H_{X\to Y}(\tau)$. \label{fig:sampleSizeHenonentropy}}
\end{figure}

\end{enumerate}
\end{document}